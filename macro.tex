%!TEX root = zomeFab.tex
%%% Packages use.
\usepackage[labelfont=bf,textfont=it]{caption}
% \usepackage{enumitem}
% \usepackage{algorithm2e}
\usepackage{comment}
\usepackage{amsmath}
\usepackage{amssymb} 
% \usepackage{amsthm}
\usepackage{amsfonts}
\usepackage{multirow}
\usepackage{subfigure}
\usepackage{color}
% \usepackage{booktabs}
\usepackage{ifthen}
% \usepackage{hyperref}
\usepackage{xcolor}% http://ctan.org/pkg/xcolor
% \usepackage{todonotes}
\usepackage[normalem]{ulem} % for sout
\newcommand{\etal}{{\textit{et~al.}}}
\usepackage{enumitem}
\usepackage[encapsulated]{CJK} 
\usepackage{wrapfig}
% \begin{CJK}{UTF8}{bsmi}                    % 開始 CJ環境,設定編碼,設定字體 

%%% Color definition.
\definecolor{gray}{rgb}{0.5,0.5,0.5}
\definecolor{green}{rgb}{0, 0.6, 0}
\definecolor{orange}{rgb}{1, 0.5, 0}
\definecolor{mahogany}{rgb}{0.75, 0.25, 0.0}
\definecolor{purple}{rgb}{0.6, 0, 0.6}
\definecolor{darkgreen}{rgb}{0, 0.3, 0}
\definecolor{orange}{rgb}{1, 0.5, 0.}

%%% Editing comments.
% ignore this
\newcommand{\ignore}[1]{}
\newcommand{\nothing}[1]{}
% comment
\newcommand{\cmt}[1]{\begin{sloppypar}\large\textcolor{red}{#1}\end{sloppypar}}
% the note in the paper
% \newcommand{\note}[1]{\cmt{Note: #1}}
% todo list
\newcommand{\note}[1]{\textcolor{red}{#1}}

%%% Editing comment.
% josh
% \newcommand{\ichao}[1]{\textcolor{blue}{#1}}
\newcommand{\ichao}[1]{\textcolor{blue}{ichao: #1}}
\newcommand{\intere}[1]{\textcolor{purple}{#1}}
\newcommand{\chinky}[1]{\textcolor{darkgreen}{#1}}
\newcommand{\chireplace}[2]{\textcolor{red}{\sout{#1}} \textcolor{darkgreen}{#2}}
%\newcommand{\chiremove}[1]{\sout{#1}}




%%% algorithm use 
% test for math env line number issue
% \newcommand*\patchAmsMathEnvironmentForLineno[1]{%
%   \expandafter\let\csname old#1\expandafter\endcsname\csname #1\endcsname
%   \expandafter\let\csname oldend#1\expandafter\endcsname\csname end#1\endcsname
%   \renewenvironment{#1}%
%      {\linenomath\csname old#1\endcsname}%
%      {\csname oldend#1\endcsname\endlinenomath}}% 
% \newcommand*\patchBothAmsMathEnvironmentsForLineno[1]{%
%   \patchAmsMathEnvironmentForLineno{#1}%
%   \patchAmsMathEnvironmentForLineno{#1*}}%
% \AtBeginDocument{%
% \patchBothAmsMathEnvironmentsForLineno{equation}%
% \patchBothAmsMathEnvironmentsForLineno{align}%
% \patchBothAmsMathEnvironmentsForLineno{flalign}%
% \patchBothAmsMathEnvironmentsForLineno{alignat}%
% \patchBothAmsMathEnvironmentsForLineno{gather}%
% \patchBothAmsMathEnvironmentsForLineno{multline}%
% }
\usepackage{algorithm}
\usepackage{algorithmicx}
\usepackage{algpseudocode}
\algnewcommand\algorithmicinput{\textbf{Input:}}
\algnewcommand\INPUT{\item[\algorithmicinput]}
\algnewcommand\algorithmicoutput{\textbf{Output:}}
\algnewcommand\OUTPUT{\item[\algorithmicoutput]}
\algnewcommand\algorithmicforeach{\textbf{for each}}
% \algnewcommand\algorithmicrepeat{\textbf{Repeat:}}
% \algnewcommand\REPEAT{\item[\algorithmicrepeat]}


\algdef{S}[FOR]{ForEach}[1]{\algorithmicforeach\ #1\ \algorithmicdo}
\algrenewcommand{\alglinenumber}[1]{\color{red!80!blue}\footnotesize#1:}
\renewcommand{\algorithmiccomment}[1]{\hfill$\triangleright$\textcolor{blue}{#1}}
\algnewcommand\Func[2]{\textcolor{green}{#1}\textcolor{green}{(#2)}}
\algnewcommand\Insert[2]{Insert {#1} to #2.}
\algnewcommand\Input[1]{\State \textbf{Input: } #1}
\algnewcommand\Output[1]{\State \textbf{Output: } #1}
% \algnewcommand\Repeat{\State \textbf{Repeat }}
% \algnewcommand\Until[1]{\State \textbf{until } #1}



%%% Frequently used terms.
% \newcommand{\etal}{{\it{et~al.}}}
\newcommand{\ie}{i.e.}
\newcommand{\eg}{e.g.}
\newcommand{\figname}{Figure}
\newcommand{\tabname}{Table}
\newcommand{\secname}{Section}
\newcommand{\algoname}{Algorithm}
\newcommand{\eqname}{Eq.}


