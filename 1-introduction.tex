%\firstsection{Introduction}
%!TEX root = zomeFab.tex
\section{Introduction}
\label{sec:introduction}

% The production cost of 3D printer has been declining and it's availability for everyone leads to the emergence of large amount of applications. 
The recent widespread of consumer\chinky{-}level 3D printer lead to the emergence of large amount academic and industrial fabrication applications.
However, the consumer-level 3D printers have several shortcomings including long printing time, limited output size and \chinky{the } high cost of materials. 
In order to address these shortcomings, many of the researches are proposed.
% There are several approaches and researches aim to solve these drawbacks. 
For saving time and materials, the modern 3D printers allow user\chinky{s} to save materials by switching between different fill-rate settings.
Meanwhile, different internal patterns are proposed~\cite{Lu:2014:BSW} to save materials while maintain\chinky{ing} the \chireplace{structure}{structural} soundness.
% provide different fill-rate settings to use less material and research like~\cite{Lu:2014:BSW} can save materials effectively. 
% The size of objects that 3D printing can produce are typically limited by 3D printers. 
For building a large-scale fabrication, lots of researches \cite{Medell:2007:ALRP, Hao:2011:APLM, Luo:2012:CPM, Hu:2014:APS, Vanek:2014:PMVO} focus on this problem and share the spirit of partitioning the object into sub-part\chinky{s} that can be fitted into printing volume.
% They all have same characteristic that partition the object to fit the size, then assembling printed parts.

% The requirements for a large-scale fabrication will enlarge the drawbacks as mentioned above.
However, the saving of time and material from existing methods are still not sufficient for large-scale shape fabrication.
% It turns out that even we employ state-of-the-art methods, time and material spending are still too high. 
\chireplace{To decrease cost extremely}{To immensely decrease the cost}, our novel idea is to combine 3D printing with another structure. 
That desire structure must meet several criteria such as (i) easy to assemble, (ii) re-usable, and (iii) robustness, so that the resulted fabricated shape is simple to build and reliable simultaneously. 
Base on our requirements, the popular modeling system, \ie~\emph{Zometool}~\cite{davis2007mathematics}, is suitable for coarse fabrication.
In addition to the above advantages, \emph{Zometool} still has several \chireplace{good}{excellent} characteristics: 
(i) structural properties, such as stability, expandability and lightness satisfying the requirements for large-scale fabrication. 
(ii) Independent structure and modularity can parallelize the construction to speed up the building process.
Hence, Zometool is a \chireplace{good}{attractive} candidate to replace solid 3D printing material as the \chireplace{inner robust}{robust inner} structure of large-scale structure.
Therefore, the goal of this work is to develop a computational method that combines Zometool structure and 3D printing to reduce the time and material costs in large-scale fabrication.
% In this paper, we present a novel method, called ZomeFab, that can automatically generate both outer shell and inner Zometool structures approximating to a given 3D input mesh.
Given the desired 3D shape, we design an optimization process to synthesize the inner Zometool structure, which balance\chinky{s} between the shape similarity and structure complexity.
We leverage Simulated Annealing to explore the huge structure space effectively.
% We first perform mesh segmentation to split the complex 3D mesh into different shape parts.
% For each part, we fit the smallest cube of Zometool structure as initial inner structure.
% From the initial Zometool structure, we use Simulated Annealing algorithm to effectively explore the huge structure space.
Next, with the optimized Zometool structure, we hollowed the shape to obtain the outer shell and partition the outer shell \chireplace{with respect to}{concerning} several criteria including simplicity and printability.
We formulate these criteria in a single MRF problem and solve it with graph-cut algorithm~\cite{boykov:2004:experimental}.
We also design a \chireplace{special}{particular} type of connectors \chireplace{and}{furthermore,} optimize their positions for assembling the inner Zometool structure and outer shell.

There are two primary contributions in this paper:
\begin{enumerate}
\item We propose an optimization framework to synthesize the inner Zometool structure to replace the solid printed materials in large-scale fabrication, which greatly reduce the printing cost including time and materials.
% Filled by Zometool structure instead of solid printed is greatly reduce the printing cost including time and materials.
\item We design and print a special connector and optimize their layout for better combining both inner Zometool structure and outer printed shell.
\end{enumerate}

% The rest of this paper is organized as follows. In  \secname~\ref{sec:relatedwork}, we surveyed several previous methods for computational fabrication and applications of modeling system Zometool.  \secname~\ref{sec:Zometool}, \secname~\ref{sec:surf_part} and \secname~\ref{sec:fab} describe the three main stages of our method: \emph{Zometool construction}, \emph{surface partition} and \emph{Fabrication}, respectively. In \secname~\ref{sec:result}, we demonstrate the results of our method.
% Finally, we make a conclusion in  \secname~\ref{sec:conclusion}.
