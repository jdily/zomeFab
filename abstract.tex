% \begin{abstract}
\abstract{
In recent years, personalized fabrication has attracted many attentions due to the widespread of consumer-level 3D printers.
However, consumer 3D printers still suffer from shortcomings such as \ignore{the relatively }long production time and limited output size, which are undesirable factors to large-scale rapid-prototyping.
In order to fabricate \chinky{a} large-scale object,\ignore{construct a large-scale fabrication, the hybrid approach is introduced.}
\ignore{In this paper, }
we propose a 3D fabrication method that combines 3D printing and Zometool structure for cost-effective fabrication of large objects.
The key of our approach is to utilize compact, \chireplace{solid}{sturdy} and re-usable internal structure (Zometool) to \chinky{infill fabrications and} replace \chiremove{the} both time and material-consuming 3D-printed materials.
\chireplace{And}{ Moreover, }as a result, we can \chireplace{greatly}{significantly} reduce the cost and time by printing \chiremove{a} thin 3D external shells only.
% The key of our approach is using Zometool as a compact and solid internal structure, and 
% to build internal structure
% then attach thin 3D-printed external shells.
We design an optimization framework to generate both Zometool structure and printed surface partitions by balancing between several criteria including printability, material saving, and Zometool structure complexity.
% We optimize these criteria and generate both the Zometool structures and the surface partitions.
We demonstrate the effectiveness of the proposed method by a variety of 3D models along with examples of the physically fabricated objects.

\noindent
{\bf Keywords:} Geometric Algorithms, Fabrication, Curve, surface, solid, and object representations.
}


% \end{abstract}